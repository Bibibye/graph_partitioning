\documentclass[12pt, a4paper]{article}
\usepackage[utf8]{inputenc}
\usepackage[T1]{fontenc}
\usepackage[french]{babel}
\usepackage{csquotes}
\usepackage{hyperref}

\begin{document}
\newcommand\Nom\textsc
\newcommand\Terme\textit

\title{Partitionnement de graphes}
\author{\Nom{Raphaël Gaudy} \and \Nom{Joris Pablo}}
\date\today
\maketitle

\begin{center}
Pour le vendredi 20 Mars 2015, 18h00
\end{center}

\section{Introduction}
Ce projet de Graphes et Recherche Opérationnelle a pour problème de décomposer des graphes en $K$ classes à peu près équivalentes en minimisant le poids des arêtes inter-classes. Cette décomposition doit être possible selon 4 méthodes différentes : énumération, descente de gradient, recuit simulé et une méta-heuristique au choix (nous avons choisi la méthode de la recherche tabou). Ce rapport explique comment utiliser notre code et détaille nos choix d'implémentation.

\section{Définition des termes du problème et utilisation du code fourni}
\'Etant donné l'énoncé du problème, il est important de définir quel est $K$ et ce que signifie ``à peu près équitable''. Nous avons fait le choix de définir $K$ à la compilation de notre exécutable (2 par défaut si rien n'est spécifié), et ``à peu près équitable'' est un prédicat évalué vrai si la différence de taille entre tout couple de classes d'une solution est au plus de 1\% (arrondi au supérieur).

Pour compiler le projet, il suffit d'utiliser :

\begin{verbatim}
make
\end{verbatim}

Ou, si on veut changer le nombre de classes (par exemple 3) :

\begin{verbatim}
make K=3
\end{verbatim}

Un exécutable \texttt{partition} est créé et propose le message d'aide suivant :

\begin{verbatim}
$ ./partition 
Usage : ./partition <heuristic> <neighborhood> <filename>

Heuristics available are :
    - explicit_enum
    - gradient_descent
    - taboo
    - simulated_annealing

Neighborhoods available are :
    - swap
    - sweep
    - pick_n_drop
\end{verbatim}

Le graphe \texttt{filename} doit être du même format que les graphes présents dans le répertoire \texttt{graphs} (celui convenu pour le projet).

Un exemple d'utilisation :

\begin{verbatim}
$ ./partition gradient_descent swap graphs/10.txt
{3 5 2 9 10}
{1 8 4 6 7}
f_opt = 13.000000
\end{verbatim}

La sortie donne donc le partitionnement optimal trouvé (ici en 2 classes) ainsi que la valeur de notre fonction à minimiser pour cette solution.

\section{\'Enumération}
Pour notre méthode par énumération, nous avons choisi une énumération explicite. De manière à pouvoir les énumérer simplement, chaque solution est représentée par un nombre en base $K$ de $n$ digits ($n$ étant le nombre de sommets du graphe). Dans une solution donnée de la forme $d_1 d_2 \dots d_n$, chaque sommet $i$ se trouve dans la classe $d_i$. Il suffit donc de compter de $0 \dots 0$ à $(K-1) \dots (K-1)$ pour passer en revu toutes les solutions (certaines sont considérées plusieurs fois par alpha-équivalence sur les classes).

Bien que la notion de voisinage n'ait aucun sens pour cette méthode, il faut quand même en spécifier un à l'exécution de notre programme (toutes les méthodes sont traitées de manière générique). Ce dernier ne sera pas utilisé.

\section{Descente de gradient}

\section{Recuit simulé}
Pour la méthode du recuit simulé, plusieurs paramètres sont à prendre en compte :
\begin{itemize}
\item La solution initiale : par simplicité, nous avons sélectionné la solution initiale au hasard.
\item La température initiale : cette dernière peut être définie à la compilation (variable \texttt{SIMULATED\_ANNEALING\_T\_0}). Par défaut, elle vaut 1000.
\item La condition de sortie de la première boucle : par défaut, on sort de la boucle lorsque la température a atteint une certaine valeur (0.1 par défaut). Cette valeur peut être changée à la compilation (variable \texttt{SIMULATED\_ANNEALING\_T\_MIN}). On peut aussi choisir à la compilation de définir une probabilité d'acceptation, auquel cas la condition sur la température n'est plus vérifiée (variable \texttt{RANDOM\_ANNEALING} à \texttt{yes} pour activer, et variable \texttt{SIMULATED\_ANNEALING\_ACCEPTATION\_PROBA} à changer pour modifier la proba. Par défaut, cette variable vaut 100, ce qui équivaut à 1 chance sur 100).
\item La condition de sortie de la deuxième boucle : on sort de la deuxième boucle lorsqu'on a sélectionné un nombre de fois donné une solution moins bonne que la solution courante (par défaut $10 \times n$ avec $n$ le nombre de sommets du graphe). Cette valeur peut être changée à la compilation (avec la variable \texttt{SIMULATED\_ANNEALING\_NB\_FAIL\_MAX} qui sera multipliée par $n$).
\item La varitation de température : nous avons choisi de suivre une suite géométrique de raison $0.9$ par défaut. Encore une fois, ce paramètre peut être changé à la compilation avec la variable \texttt{SIMULATED\_ANNEALING\_RATIO}.
\end{itemize}

Un exemple de compilation qui change certains paramètres du recuit simulé :
\begin{verbatim}
make RANDOM_ANNEALING=yes SIMULATED_ANNEALING_ACCEPTATION_PROBA=1000
     SIMULATED_ANNEALING_RATIO=0.8
\end{verbatim}

\'Etonamment, notre recuit simulé semble ne pas fonctionner avec le voisinnage pick'n drop sur des graphes avec un nombre de sommets impair.

\section{Recherche tabou}
La recherche par tabou, suivant les recommandations du cours, reste
simple pour éviter les complications : la liste des tabous à chaque itération est stockée 
dans une file dans laquelle on pousse les voisins examinés 
à chaque itération (en retirant le plus ancien s'il n'y a pas de place).

Il y a donc deux paramètres :
\begin{itemize}
\item \texttt{TABOO\_{}SIZE}, par défaut à 10 : c'est la taille de la liste d'exclusion.
\item \texttt{TABOO\_{}ITERMAX}, le nombre maximal d'itération, par défaut 100.
\end{itemize}
Comme avec les autres méthodes, ces paramètres sont changeables à la compilation.

\section{Résultats}

\end{document}

%%% Local Variables: 
%%% mode: latex
%%% TeX-master: t
%%% End: 
